\newpage

\section{Problem formulation}

When performing on a stage, proper handling of light is important. This can be achieved through controllable spotlights. They have to be able to rotate horizontally and vertically, thereby targeting specific points on the stage and do so autonomously.\\
The Pan \& Tilt System (P \& T System) can be made to meet these requirements.


\subsection{Definition of goals}
The main goals of the product is, to be able to point the cone of light from the spotlight to a specific point on a predefined surface, or follow a human walking around on a surface. The user should have the option to specify the limits of the spotlights working space. The positions or paths should be set by the user and the movement of the cone should be pleasant to watch.

\subsection{Requirements specification}
\label{sec:RequirementsSpecification}

The requirements will be divide into primary and secondary requirements, so as to identify the most important objectives.


\subsubsection{Primary requirements}
\label{sec:Primaryrequirements}

% Requirements to the hardware
\textit{Hardware requirements}

\begin{itemize}

\item The controller has to be implemented on the FPGA(Field-programmable gate array).

\item The communication between the microprocessor and the FPGA has to be done with SPI(Serial Peripheral Interface Bus).

\item The FPGA has to control the PWM(Pulse-width modulation) signal for the motors.
\end{itemize}


% Requirements to the p&t driver 
\textit{Driver requirements}
\begin{itemize}
\item The system cannot have an overshoot of more than 10\%.

\item An increment in the hall sensor counter must have a settling time of at least 50 mS. 


\item The spotlight should have a steady state close to zero
\end{itemize}


% requirements to the applicaiton
\textit{Application requirements}
\begin{itemize}

\item The application must have support for the different spotlight height configurations.

\item The application must support different stage dimensions, to a maximum of 14 m x 14 m. 

\end{itemize}


\subsubsection{Secondary requirements}
These are the requirements that would be nice improvements to the project, but are not strictly necessary for the successful completion of the project. 

\begin{itemize}
\item The movement between two points should be in a smooth motion.

\item The P\&T system is to recalibrate the position of the Pan and the tilt during startup.

\item The spotlight can be controlled by an application.

\item The spotlight can be controlled by a joystick.

\item At a scene of dimension 5 m x 5 m, from a height of 5 meters, the system must achieve a velocity of at least $2.5 m/s^{1}$ relative to the scene, which is the maximum velocity the average person would walk. 
\footnote{Wikipedia, Preferred walking speed, \url{https://en.wikipedia.org/wiki/Preferred_walking_speed}, 23/05/16}

\end{itemize}
