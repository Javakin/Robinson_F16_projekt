\newpage
\section{Project Management}

Når et projekt når en vis størrelse, bliver det nødvendigt med projektstyring.\\
Til dette formål er der arbejdet med en agil tilgang.\\

Der er blevet benyttet flere agile metoder.

\subsection{Scrum}

Scrum er en agil møde- og organiseringsmetode, der er beregnet til små grupper. Et normalt møde er delt op i flere punkter og hvert medlem af holdet, gennemgår på tur alle punkter.

\subsubsection{Mødepunkter}
\begin{itemize}[noitemsep]
	\item Hvad har du lavet siden sidste scrum møde.
	\item Hvad vil du lave til næste gang?
	\item Er der nogle åbenlyse ting der forhindrer dig i at nå vores mål?
	\item Er der nogle nye ukendte opgaver der skal tilføjes til Sprint Backlog?
	\item Har du lært noget nyt der kan være brugbart for de andre gruppemedlemmer?
\end{itemize}

Ideelt set holdes mødet på samme tid hver dag, også selv om der skulle mangle nogen fra holdet.
Det er forsøgt at begrænse de enkelte møder til 20 min. Andre ting som skal tages op, bliver løbende noteret, så det kan diskuteres efter mødet. Dette resulterer i, at mødet kan holdes kort og præcist, så det ikke tager for meget arbejdstid fra folk, der ikke skal bruge det til noget.\\

Under hvert møde, blev der udarbejdet et mødereferat, hvor hver persons svar til de forskellige punkter, blev skrevet ind. Bilag \ref{itm:Modereferat1}


\subsubsection{Effekt}
Hele holdet får en god idé om hvor projektet er på vej hen, samt hvad de øvrige medlemmer laver.
Eventuelle problemer bliver bragt frem, og ny viden der er relevant kan blive delt med alle.\\
Det giver et godt overblik og sammenhæng mellem hvad der bliver arbejdet på. Der har dog under projektet nok været brugt for meget tid på Scrum møder i forhold til den gavnlige effekt det har givet.


\subsection{Pair programming}

Pair programming går som navnet antyder, ud på at to personer programmerer sammen på en computer. Det kan virke som et spild af arbejdskraft, da der jo kun kan være en der koder af gangen, men det kan hjælpe til at give noget bedre kode. Grunden er at der hele tiden vil være en til at se hvis der bliver laver fejl, samt komme med gode ideer. Det er også en god måde at dele erfaring og viden om projektet.\\

Det er delvis blevet benyttet, men der har også været en del individuel programmering.

\subsubsection{Effekt}
Det har givet en bredere og bedre forståelse for koden, og har forhindret nogle fejl som måske ellers først ville blive fanget senere.

\subsection{Iterativ og inkrementel udvikling}
Der eksisterer mange metoder til at strukturere udviklingen af et program. som f.eks. "Vandfaldsmetoden". I denne metode, laver man hele programmet ud i et, fra start til slut og der bliver ikke gået tilbage for at lave noget om. Det er ikke en fleksibel måde at udvikle et program på, hvor det vil være meget svært at tilpasse sig ændringer i kravspecifikationerne. Det kræver også at der er fuldstændig styr på hvad der skal laves og hvordan.\\

For at undgå de svagheder, er der til dette projekt valgt en iterativ og inkrementel tilgang.\\
Hele projektet er delt op i flere dele. Hver del er gradvist udviklet og ændringer i en del har en effekt på de andre. Der er lavet test løbende og delene er blevet tilpasset de nye erfaringer der er blevet gjort.


\subsubsection{Effekt}
Da der løbende har været tests af de enkelte dele og hvordan de spiller sammen, har det været muligt at lave ændringer efterhånden som det har været nødvendigt.\\

%Er Tidsplan delen nødvengig

\subsection{Tidsplan}
For at holde styr på hvad der skulle laves hvornår og hvor meget der manglede, blev der lavet en tidsplan. Denne blev drøftet ved flere møder, og opdateret efterhånden som projektet skred frem.

\subsubsection{Effekt}
En dynamisk tidsplan giver et godt overblik over hvor projektet er på vej hen, samtidig med at der kan tages hensyn til de ændringer som skulle opstå.  Bilag \ref{itm:Tidsplan1}

\subsection{Delkonklusion}
Det har været et godt valg, at bruge agile metoder, som grund til organiseringen af projektet, men der er både fordele og ulemper.\\

%\textbf{Fordele}
\begin{itemize}
	\item Fordele
\begin{itemize}
	\item Pair Programming har givet en mere effektiv og mere fejlfri kode.
	\item Vidensdeling er en af de centrale roller, og gruppemedlemmerne har en god idé om hvad der er blevet skrevet på 					  projektet.
	\item Alle gruppemedlemmer får mulighed for at få nye opgaver ved hvert scrum møde.
	\item De agile arbejdsmetoder har givet et godt flow i projektarbejdet og gruppemedlemmerne er kommet 					  hinanden mere ved, i forhold til et normalt plan-baseret projekt.
	\item Når alle som er involveret i projektet er til stede, bliver det mere overskueligt at tilpasse sig ændringer.
\end{itemize}

%\textbf{Ulemper}
	\item Ulemper
\begin{itemize}
	\item Det var svært at holde et lydniveau der var acceptabelt til at arbejde i, når hele gruppen arbejder i samme rum.
	\item Scrum møder tog typisk længere end 20 min, og der har været meget spildtid grundet den store mængde personer til hvert møde.
	\item Scrum møder kan tage unødvendig tid, som kan have bruges til at arbejde.
	\item Scrum kræver en stor disciplin, og hvis den ikke er til stede, skrider tidsplanen for scrum mødet.
\end{itemize}
\end{itemize}

\subsection{Evaluering}
I fremtidige projekter, kan projektstyringen potentielt forbedres på følgende måder: \\
\begin{itemize}[noitemsep]
	\item Have en mere effektiv tilgang til scrum-møder, hvor kun emner som er nødvendige for projektets fremskriden diskuteres.
	\item Afsætte tid til individuelt arbejde, hvor de enkelte gruppemedlemmer har ro til fordybelse.
	\item Være mere konsekvent med fælles afholdelse af pauser, således at et fornuftigt arbejdsklima kan bevares, til trods for en stor mængde samlede mennesker.
	\item Fortsat benytte de agile værktøjer, for at genskabe den høje mængde af vidensdeling og fælles indsforståethed med projektets mål og udfordringer.
\end{itemize}