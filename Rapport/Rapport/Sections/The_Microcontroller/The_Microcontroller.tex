\section{The Microcontroller}
\label{sec:TheMicrocontroller}

 Given the design requirements in section \ref{sec:RequirementsSpecification}, and the block diagram in section \ref{sec:TheSystem}. This includes all the functionality that is to be placed in the microcontroller. From here the next stage is to define the content of these blocks, and how they interact with each other. 
 
 The microcontroller is to establish an application programmer interface (API) to the P\&T driver placed in the FPGA. The microcontroller will also run the application that uses this interface in order to help the user achieve his/her goal, which is to control the spotlight.
 
 The program will be written in the C language, and it will be documented using task diagrams and state machines. 


% program structure
\subsection{Choice of Operating System}
\label{sec:ChoiceofOperatingSystem}

It is safe to assume that because this program will be using more than one task, some sort of operating system (OS) is required. The decision of which operating system to use, will have great effect on the entire project, since it defines the domain in which the entire application will have to operate in. This OS must therefore as a minimum support the ARM Cortex M-4 processor, which is used in this project. 

By using a Real Time Operating System (RTOS) like freeRTOS, it would give the advantages of preemptive scheduling. Being able to preempt a low priority task to pave the way for a task of higher priority, is an advantage when low response time is important. However. the application will only benefit from using a RTOS if its size extends 1 MB. Application smaller then 64KB simply would not be large enough for a RTOS to have any effect relative to a standard OS.\footnote{Texas Instruments, Why use RTOS in MCU Applications, \url{http://www.ti.com.cn/cn/lit/wp/spry238/spry238.pdf}, 23/05/16}

Assuming that the project will reach this size, combined with the fact that this OS is used in one of the courses doing this semester, freeRTOS was chosen.

\subsection{Program Structure}
\label{sec:ProgramStructure}

Given the above mentioned requirements, a task diagram for the system was created, here the program were to handle two different sources of user input, in a manner that would not effect the application. 

Insert task diagram here 


walk through the system, separating the program into tasks starting from the block diagram. ( see the criteria in Morten's ppt. )

one task one input queue

what drivers where required to achieve the goal of the program,

now add the used semaphores at critical sections

Now when the task diagram has been made, it is time to considerer possible structural weakness such as deadlocks and handling og queues. 



\subsection{The Application Programmer Interface}
\label{sec:TheApplicationProgrammerInterface}

\subsection{Event Queues and Shared State Memory}
\label{sec:EventQueuesandSharedStateMemory}

it will ensure that only events can be found in the queue, data will be saved and an event stating that data can be read will be send enstead

this is so that the data never will look like an event.

How is this implemented


\subsection{Deadlock Protection}
\label{sec:DeadlockProtection}

A deadlock is when one or more process cannot proceed because some of the resources it need is held by another process. The two processes will "compete" with each other and neither of them will finish.
 
\begin{enumerate}[noitemsep]
	
	\item \textbf{Mutual exclusion} is the concept of making sure that two or more concurrent processes are not in their critical section at the same time. The critical section is a part of the program that cant be used by more than one process at the time.\\ If more than one process is in the same critical section at the same time errors are bound to happen.
	
	\item \textbf{Hold and wait} is when a process holds on to a resources and wait for the rest of the resources it needs.
	
	\item \textbf{no preemption} means that only the process that holds the resource can release it. If preemption is allowed, the program is able to take resources from a process and allocate it to another process.
	
	\item \textbf{Circular wait} is when one process is waiting for at resource that another process is has and the other process is waiting for a resource held by the first one. It could also be a chain of processes where every process needs a resource held by the next and the last one needs a resource held by the first.
	
\textbf{(add source XXXX)}
	
\end{enumerate}

All four conditions needs to hold true in order for a deadlock to occur. You can therefore provide a deadlock-free environment by avoiding one of the conditions. 

To avoid deadlocks in the system, the circular wait condition is removed via semaphores. This insures that deadlock cannot happen. In figure \ref{fig:Semaphore} a code example of how it is used.

\begin{figure}[h!]
\centering
\includegraphics[scale=0.6]{Billeder/Micro_Controller/Semaphore_code_example.png}
\caption{ Semaphore code example }
\label{fig:Semaphore}
\end{figure}

First the process takes one semaphore. Then it takes another, do \textbf{SKRIV HVAD DEN GØR}. After that it releases the first semaphore and then the last. This insures that it has all the resources it needs to do its work and no deadlock is possible.













\subsection{SPI-Master}
\label{sec:SPIMaster}

The microcontroller have to share information with the FPGA, and according to the requirements mentioned in section \ref{sec:Primaryrequirements}, this communication has to happen using Serial Peripheral Interface (SPI). The protocol will operate as descriped in section \ref{sec:SPIcommunication}. It is therefore important that this link never will be a bottleneck for the system application. 

The requirements of the minimum bit-rate depends on what data that needs to be send, and how often it needs to send it. Since the PID-controller is placed in the FPGA, the application only needs to update the current position once every 5 ms, which is the minimal delay that freeRTOS can put on a task. 

The length of the messages sent is defined by the driver placed in the FPGA. As described in section \ref{sec:Implementation} all messages will have the length of 14 bits. Updating both pan and tilt requires two messages. Considering all other messages but the main pulling sequence as negligible results in a total bitrate of:

\begin{equation}
Bitrate = \frac{
14 bit * 2	
}{
0.05s
} = 5.600 bit/sec 
\end{equation}

The Cortex M4 has a built-in freescale spi module, that supports four different variations of spi, see appendix \ref{sec:CortexM4Datasheet}. However, implementing the spi from scratch, gives the highest degree of control over the process, which is preferable in this project. 

% full dreasd task diagram
\begin{figure}
	\centering
	\includegraphics[scale = 0.7] {Billeder/SPI-master}
	\caption{The SPI-master state machine}
	\label{fig:SPI-master}
\end{figure}

The state machine for the task is illustrated in figure \ref{fig:SPI-master}. It consists of three states and is initiated waiting for a message to transmit from the spi\_tx\_queue. When a message is pulled from the queue, it jumps to the send state, and after sending the message, the received message will be handled by the P\&T API in the Receive task. From here it will only switch back to the idle state if the API call was successful, else it will continue until it succeeds. 


\subsubsection{Performance Test} 
\label{sec:PerformanceTest}
After implementing the the SPI-master task in the final application, the max bit-rate was tested by overloading the spi\_tx\_queue, and giving the SPI-master task High priority in the OS. This made the the task perform at maximum capacity. The measurements of the SPI connection can be seen in figure \ref{fig:HightPerformance}. Since it is able to transmit messages of 14 bots at 7.7 kHz the bitrate will be the product of these two values giving the result $107 kbit/sek$


\begin{figure}
	\centering
	\includegraphics[scale = 0.7] {Billeder/HightPerformance}
	\caption{Shows the max performance of the SPI-Master task}
	\label{fig:HightPerformance}
\end{figure}
 

\subsection{The Application}
\label{sec:TheApplication}

\subsubsection{The Kernel task}
making the state mashine for the kernel-task

separating commands in groups depending on the number of parameters to use. 

the list of commands that will be supported (se the list on google drive XXXX)

\subsubsection{The light-show task}
making the state machine for the light-show-task








% summery 
\subsection{Overview}

The functionality of the system was meant as a fully dressed system containing all the features that would be needed while working with a real spotlight. However some of the secondary requirements cut from the final project.

\textbf{SKRIV GRUND SO DET IKKE LYDER SOM EN UNDSKYLDNING}


\textbf{The joystick:} Being able to control the spotlight via a PS2 controller is a secondary requirement. A working driver was made but the corresponding API was not. This meant that this feature was not implemented. This is not a problem as it is possible to control the system via the a PC though the UART.


\textbf{Position vs given coordinates:} The input from the P\&T system is tachs(falling and rising edges from the Hall effect sensors). This correspond to an orientation on a sphere. As the target for the spotlight is a flat surface, some conversions is needed in order to translate the desired position on the scene to an amount of tachs. The conversion takes place in \textbf{HVOR SKER DET?}.


\textbf{Following a walking person:} The system is not build with the option to change the speed, based on user input, between points. This means that rather than following a walking person, the system points at different points of the stage.


This is the functions that did not make it to the implementation

\begin{enumerate}[noitemsep]
	
	\item PS2 controller and its corresponding api
	
	\item Calculating the tack-position from the relative coordinates given by the application  
	
	

\end{enumerate}