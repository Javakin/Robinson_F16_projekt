\subsection{Deadlock Protection}
\label{sec:DeadlockProtection}

A deadlock has happened when two or more processes wait for each other to release a resource, but since they all are waiting, no one will ever finish their task. The tasks and resources will therefore be unusable or deadlocked for the rest of the applications lifetime. 


\begin{enumerate}[noitemsep]
	
	\item \textbf{Mutual exclusion}: is the concept of making sure that two or more concurrent processes are not in their critical section at the same time. 
	
	\item \textbf{Hold and wait}: is when a process holds on to a resources and wait for the rest of the resources it needs.
	
	\item \textbf{no preemption}: means that only the process that holds the resource can release it. If preemption is allowed, the program is able to take resources from a process and allocate it to another process if needed.
	
	\item \textbf{Circular wait}: is when one process is waiting for at resource that another process have, and the new process is waiting for a resource held by the first one. As long at this makes up a closed chain of waiting processes there is a circular wait.

	
\end{enumerate}

All four conditions need to hold true in order for a deadlock to occur. You can therefore provide a deadlock-free environment, by avoiding one of these conditions\footnote{Operating system Concepts 9th edition, by Abraham Silbershatz, page 319}. 

To avoid deadlocks in the system, the circular wait condition is removed via semaphores. However since the system only have one semaphore circular wait will never be an option. 
