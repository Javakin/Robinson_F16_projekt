\subsection{Choice of Operating System}
\label{sec:ChoiceofOperatingSystem}

It is safe to assume that because this program will be using more than one task, some sort of operating system (OS) is required. The decision of which operating system to use, will have great effect on the entire project, since it defines the domain in which the entire application will have to operate in. This OS must therefore as a minimum support the ARM Cortex M-4 processor, which is used in this project. 

By using a Real Time Operating System (RTOS) like freeRTOS, it would give the advantages of preemptive scheduling. Being able to preempt a low priority task to pave the way for a task of higher priority, is an advantage when low response time is important. However. the application will only benefit from using a RTOS if its size extends 1 MB. Application smaller then 64KB simply would not be large enough for a RTOS to have any effect relative to a standard OS.\footnote{Texas Instruments, Why use RTOS in MCU Applications, \url{http://www.ti.com.cn/cn/lit/wp/spry238/spry238.pdf}, 23/05/16}

Assuming that the project will reach this size, combined with the fact that this OS is used in one of the courses doing this semester, freeRTOS was chosen.