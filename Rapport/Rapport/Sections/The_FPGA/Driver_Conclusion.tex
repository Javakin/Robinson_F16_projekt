\section{Driver conclusion}

Even though the driver runs quite smooth some changes could have been made in order to make it better. One of these changes would be to make a controller that actually follows the control theory for a PI controller a little better. The reason why this was not done from the start, was because of a lack of understanding how a controller could be implemented. As an example it was discovered rather late in the development stage that the optimal way of implementing the PI controller would have been by making a filter. This was however discovered too late to change. A filter was created, but an unknown error shut down the idea of implementing it as a filter since the time did not allow it to be completed. However if more time was given implementing the controller as a filter would be one of the upgrades that would make the product easier to tune because the filter follows control theory closer then the implemented controller.
Another thing would be to implement the controller as a PID controller instead since this would make a smoother controller. The control theory section also derives that the optimal controller would be the PID. The easiest way to implement the D part would be to make a speed sensor, this was however discovered late in the development stage after a failed attempt at implementing the D part through standard derivative mathematics.
An error was also found in the link between the controller block and the motor driver block since the controller is actually based on controlling a voltage rather than the duty cycle. This  error can be fixed by making a conversion from voltage to duty cycle in the controller block.
